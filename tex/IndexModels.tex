\documentclass[a4paper]{article}
\usepackage[utf8]{inputenc}
\usepackage[T1]{fontenc}
\usepackage[light,condensed,math]{kurier}
\usepackage[ngerman]{babel}
\usepackage{ntheorem}
\usepackage{graphicx}
\usepackage{floatrow}
\usepackage{float}
\usepackage{hyperref}
\usepackage{mathtools}
\usepackage{amssymb}
\theoremstyle{break}

\newcommand{\ms}{$\mu$-$\sigma$}
\newcommand{\msd}{\ms-Diagramm}
\newcommand{\mbd}{$\mu$-$\beta$-Diagramm}

\newtheorem{defi}{Definition}[section]
\newtheorem{ann}{Bemerkung}[section]
\newtheorem{der}{Folgerung}[section]
\newtheorem{ex}{Beispiel}[section]
\newtheorem{why}{Vorteile}[section]
\newtheorem{whynot}{Nachteile}[section]


\title{Indexmodelle}
\author{Adrian E. Lehmann}

\begin{document}
        \maketitle
        \tableofcontents
        \newpage
        
\section{Single-Index-Modell (Sharpe)}
Idee: $\tilde{r}_j = \alpha_j + \beta_j \cdot \tilde{r}_I + \tilde{\epsilon}_j$\\
wobei:\\
$\alpha_j$ Überrendite zu Markt\\
$\beta_j$ Empfänglichkeit für Marktschwankungen\\
$\tilde{\epsilon}_j$ Zufälliger Störterm\\

\begin{ann}[Grundannahmen]
    Keine Systematische Verzerrungen der Störterme
        $$E(\tilde{\epsilon}_j) = 0 ~ \forall j$$
    Störterme nicht mit Marktrendite korreliert
        $$cov(\tilde{\epsilon}_j, \tilde{r}_I) = 0 ~ \forall j$$
    Störterme paarweise unkorelliert:
         $$cov(\tilde{\epsilon}_j, \tilde{\epsilon}_k) = 0 ~ \forall j \neq k$$
\end{ann}
\begin{defi}[Defensive / Aggressive Aktien]
    \begin{center}
        Aktie $i$ defensiv $\Longleftrightarrow ~ \beta^{2}_i < 1$\\
        Aktie $i$ aggressiv $\Longleftrightarrow ~ \beta^{2}_i > 1$
    \end{center}
\end{defi}
\begin{ann}[Risiko]
    2 Arten von Risiko:\\
    Systematisches Risiko $\beta_j$ (\textbf{nicht diversifizierbar})\\
    Residualrisiko (unsystematisches) $\alpha_j$ und $\tilde{\epsilon}_j$ \textbf{diversifizierbar} 
\end{ann}
\subsection{MultiIndex}
Mehr Indices um Brancheneffekte zu zeigen und $\tilde{\epsilon}$-Korrelationen zu senken.

\section{CAPM}

\textbf{Annahmen:}
\begin{itemize}
    \item Investoren optimieren auf $\mu$-$\sigma$ Basis
    \item Investoren haben homogene Erwartungen
    \item Der Anlagehorizont aller Investoren ist gleich
    \item Kapitalmärkte sind vollkommen und friktionslos
    \item Es besteht risikolose Anlage- und Aufnahmemöglichkeit zum selben Zinssatz $r$ und in bel. Höhe
    \item Märkte bestehen aus handelbaren WPs mit festem Angebot
\end{itemize}

\textbf{Input:}
\begin{itemize}
    \item Erwartungswerte über zukünftige Kurse
\end{itemize}

\textbf{Output:}
\begin{itemize}
    \item Heutige Kurse passen sich bis zur Markträumung an
\end{itemize}

\textbf{Stärken CAPM:}
\begin{itemize}
    \item Relativ einfaches linears Gleichgewichtsmodeel
    \item Plausible Annahmen
    \item Robust gegenüber aufweichen der Annahmen:
    \begin{itemize}
        \item LVK Beschränkung ändert GGW Modell nicht
        \item Risikoloses Instrument: Setze $\beta = 0$        
    \end{itemize}
\end{itemize}

\textbf{Schwächen CAPM:}
\begin{itemize}
    \item Empirisch schwer zu testen (Rolls Kritik):
    \begin{itemize}
        \item Marktportfolio unmöglich zu erreichen
        \item Lineare Risiko/Rendite Beziehung ist mathematische Tautologie
    \end{itemize}
\end{itemize}

Die Idee ist, dass jeder Investor selbst sich Aktien und risikoloses Instrument. Dies erzeugt ein \textbf{Gleichgewicht}: Der Markt ist geräumt und alle Investoren sind befriedigt. \textbf{$\Longrightarrow$ Jeder Investor hat gleiches Portfolio (genannt: ``Marktportfolio'')}

\begin{defi}{Kaptialmarktlinie - KML}
    Alle effiziente PFs liegen auf der KML (Gerade).\\
    Das einzige eff. reine Aktienportfolio ist das Marktportfolio
\end{defi}

\begin{defi}{Marktpreis des Risikos}
    $$\frac{\mu_M - r}{\sigma_M}$$
\end{defi}

\begin{der}
    Aktive Handelsstrategie NICHT sinnvoll, sondern nur das Nachbilden des Marktportfolios    
\end{der}

\begin{ann}[Risiko]
    Risiken können in 2 Klassen unterteilt werden:\\
   \begin{enumerate}
       \item Systematisches Risiko (nicht div.bar): Risikobeitrag zum Gesamtrisiko $cov(\tilde{r}_j, \tilde{r}_M)$
       \item Unsystematisches Risiko (div.bar): Unternehmensbez. Risiko $\sigma^{2}_{\epsilon_j}$
   \end{enumerate}    
\end{ann}

\begin{defi}[$\beta_j$]
    $$\beta_j \coloneqq \frac{cov(\tilde{r}_j, \tilde{r}_M)}{\sigma^{2}_M}$$
\end{defi}

\section{APT}

Seien $I_k, k \in \{1, \dots, K\}$ Faktoren

Prinzip: Keine \ms-Optimierung, sondern renditegenerierender Prozess mit
$$\tilde{r}_i = \alpha_i + \sum_{k=1}^{K=0} (\beta_{i,k}\tilde{I}_k)+ \tilde{\epsilon}_i$$

Unterstelle Markt mit Eigenschaften
\begin{enumerate}
   \item $\sum_{i=1}^{N} w_i = w^T \cdot 1_{N} = 0$ - Kein Kaptialeinsatz
   \item $\sum_{i=1}^{N} w_i \cdot \beta_{i,k} = w^T \cdot \beta_k = 0, \forall k \in \{1, \dots, K\}$ - Kein Faktorrisiko
   \item $\sum_{i=1}^{N} w^{2}_i \sigma^{2}_{\epsilon_i} \approx 0$ - Wohl diversifiziert
\end{enumerate}

\begin{der}[Linearität]
    $$\exists \lambda_i, i \in \{0, \dots, K\}:  \mu_i = \lambda_0 + \sum_{k=1}^{K}\lambda_k \cdot \beta_{i,k}$$
    Wobei $\lambda_0 = r$ \& $\lambda_k = \mu_{I_k} -r$ (Überschussrendite gegen PF, dass nur gegen $I_k$ sensitiv ist)
\end{der}

\begin{ann}
  APT gibt keine Hinweise über Anzahl und Interpretation der Faktoren
\end{ann}

\begin{ann}[APT $\leftrightarrow$ CAPM]
    \textbf{APT steht nicht im Widerspruch zu CAPM}
\end{ann}
\end{document}
