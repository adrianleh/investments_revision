\documentclass[a4paper]{article}
\usepackage[utf8]{inputenc}
\usepackage[T1]{fontenc}
\usepackage[light,condensed,math]{kurier}
\usepackage[ngerman]{babel}
\usepackage{ntheorem}
\usepackage{graphicx}
\usepackage{floatrow}
\usepackage{float}
\usepackage{hyperref}
\usepackage{mathtools}
\usepackage{amssymb}
\theoremstyle{break}

\newcommand{\ms}{$\mu$-$\sigma$}
\newcommand{\msd}{\ms-Diagramm}
\newcommand{\mbd}{$\mu$-$\beta$-Diagramm}

\newtheorem{defi}{Definition}[section]
\newtheorem{ann}{Bemerkung}[section]
\newtheorem{der}{Folgerung}[section]
\newtheorem{ex}{Beispiel}[section]
\newtheorem{why}{Vorteile}[section]
\newtheorem{whynot}{Nachteile}[section]


\title{Performancemessung}
\author{Adrian E. Lehmann}

\begin{document}
        \maketitle
        \tableofcontents
        \newpage
        
\section{Grundlagen}
    Problem: Erreichte Rendite könnte auf Grund von Glück, zu hoher Risiken oder auf Können beruhen. Wie kann man sicherstellen, dass es Können ist?
    \textbf{Analysiere: Risikoadjungierte Renditen auf Basis großer Zeiträume.}\\

2 Möglichkeiten:\\
\begin{enumerate}
    \item Vergl. PF mit \textbf{Benchmark-PF}, welches \textbf{ähnliches Risiko} hat
    \item Vergl. mit \textbf{theoretischen Bench.-PFs.}, die sich aus z.B. \textbf{CAPM oder APT ergeben} $\Rightarrow$ \hyperref[perf]{Maße}
\end{enumerate}

\section{Performance-Maße}
\label{perf}
    All folgenden Maße beruhen auf Idee, dass Kurse gemäß CAPM sich auf Basis von öffentlichen Informationen bilden und Fondsmanagers \textbf{Selectivity-Vermögen} gemessen werden kann
\subsection{Sharpe-Ratio}
    Das Sharpe Ratio stellt die Mittlere Überrendite in Relation zu dem eingegangen Gesamtrisiko:
    $$s = \frac{\bar{\mu} - r}{\sigma}$$
    \begin{der}
        Sharpe-Ratio misst Steigung der Geraden die durch risikoloses Instrument und PF verlaufen im \msd \& vergleicht mit Steigung ``risikolos $\rightarrow$ Markt-PF''
    \end{der}
\subsection{Jensen's Alpha}
    Idee: Vertikaler Abstand von erzielter mittlerer Rendite von der Gleichgewichts Rendite gemäß eigegangenem Risiko auf WPML ($\rightsquigarrow$ \mbd).
    $$\alpha = \bar{\mu} - r - (\mu_M - r)  \cdot \beta$$ 
    \textbf{Achtung:} Jensen's Alpha berücksichtigt NICHT, dass Risiko auch durch Mischung aus riskant und risikolos erz. werden kann
\subsection{Treynor-Maß}
    Das Treynor-Maß stellt die Mittlere Überrendite in Relation zu dem eingegangen systematischen Risiko:
    $$z = \frac{\bar{\mu} - r}{\beta}$$
    \begin{der}
       Treynor-Maß misst Steigung der Geraden die durch risikoloses Instrument und PF verlaufen im \mbd \& vergleicht mit Steigung ``risikolos $\rightarrow$ Markt-PF''
    \end{der}
    \begin{ann}[Risikolose Mögl.]
        Wenn zusätzlich zu Anlage Geld risikolos angelegt oder aufgenommen werden kann, dann verwende Treynor-Maß, da die Linie exakt diese Möglichkeiten abbildet.
    \end{ann}

%TODO: Slide 188
\end{document}
