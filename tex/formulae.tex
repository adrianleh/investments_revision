\documentclass[a4paper]{article}
\usepackage[utf8]{inputenc}
\usepackage[T1]{fontenc}
\usepackage[light,condensed,math]{kurier}
\usepackage[ngerman]{babel}
\usepackage{hyperref}
\usepackage{amssymb}
\usepackage{mathtools}
\usepackage{etoolbox}

\usepackage{ntheorem}
\theoremstyle{break}
\newtheorem{formula}{Formel}[section]
\newtheorem{ann}{Bemerkung}[section]
\newtheorem{der}{Folgerung}[section]

\newcommand{\st}{\sum_{t=1}^{T}}

\title{Formelsammlung}
\author{Adrian E. Lehmann}

\begin{document}
    \maketitle
    \tableofcontents
    \newpage
    \section{Grundlagen}
    Gegeben Investment $I$
    \begin{formula}[Steigung RIK]
        $$m_{RIK} = -\frac{C_1(I)}{C_0(I)}$$
    \end{formula}
    \begin{der}[Geradengleichung]
        $$C_1(I) = m_{RIK} \cdot (I_{max} - C_0(I))$$
    \end{der}
    \section{Portfoliotheorie}
    \subsection{Markowitz}
    Betrachte Aktie mit Kursen $S(t), t \in \mathbb{N}_0$ zur Zeit $t$ und mit Dividenden $D$
    \begin{formula}[Rendite]
       $$\tilde{r}=\frac{\tilde{S}(1)-S(0)+D}{S(0)}$$
    \end{formula}
    Sei $w = (w_1, \dots, w_n)$ PF aus Aktien $\{1, \dots, n\}$ mit erw. Renditen $\mu_i$ und Varianzen  $\sigma^{2}_i$
    \begin{formula}[Erw. PF-Rendite]
        $$\mu_w = \sum_{i=1}^{n}w_i\cdot\mu_i$$
    \end{formula}
    \begin{formula}[PF-Varianz] 
        $$\sigma^{2}_w = \sum_{i=1}^{n}\sum_{j=1}^{n} w_i \cdot w_j \cdot Cov(\tilde{r}_i, \tilde{r}_j)$$
        $$=  \sum_{i=1}^{n}\sum_{j=1}^{n} w_i \cdot w_j \cdot \rho_{ij} \cdot \sigma_i \cdot \sigma_j$$
    \end{formula}
    Sei $n = 2$
    \begin{formula}
        $$w_1 = 1 - w_2$$
    \end{formula}
    \begin{formula}[Erw.-Wert]
        $$\mu_w = w_1 \cdot \mu_1 + w_2 \cdot \mu_2$$
    \end{formula}
    \begin{formula}[Varianz]
        $$\sigma^{2}_w = w^{2}_1 \cdot \sigma^{2}_1 + w^{2}_2 \cdot \sigma^{2}_2 + 2 \cdot w_1 \cdot w_2 \cdot \sigma_1 \cdot \sigma_2 \cdot \rho_{12}$$
        $$\Rightarrow \sigma^{2}_w = w^{2}_1 \cdot \sigma^{2}_1 + (1-w_1)^{2} \cdot \sigma^{2}_2 + 2 \cdot w_1 \cdot (1-w_1) \cdot \sigma_1 \cdot \sigma_2 \cdot \rho_{ij}$$
    \end{formula}
    \begin{formula}[GVMP]
        $$\frac{\partial\sigma^{2}_w}{\partial{}w_1} = 2w_1 \cdot \sigma^{2}_1 - 2(1-w_1) \cdot \sigma^{2}_2 + 2 \cdot (1 - 2w_1) \cdot \sigma_1 \cdot \sigma_2 \cdot \rho_{12}$$
        $$\Rightarrow w^{GVMP}_1 = \frac{\sigma^{2}_2 - \sigma_1 \cdot \sigma_2 \cdot \rho_{12}}{\sigma^{2}_1 + \sigma^{2}_2 - 2 \cdot \sigma_1  \cdot \sigma_2 \cdot \rho_{12}}$$
    \end{formula}
    \begin{formula}[Feldman / Reisman]
        Sei $V$ die Varianz-Kovarianzmatrix
        $$\mu_w = r + w^T\mu_x$$
        $$\sigma^{2}_w = w^T \cdot V \cdot W$$ 
        \textbf{Effiziente Portfolios:}
        $$w_eff = V^{-1}\mu_x$$
        \textbf{Effiziente Linie:}
        $$\mu_w  = r + \frac{\mu_a - r}{\sigma_a} \cdot \mu_w$$
        \centering{$\Rightarrow \tilde{w} = V^{-1} \cdot \mu_x$ und $w = \frac{1}{|\tilde{w}|_1} \cdot \tilde{w}$}
    \end{formula}
    \subsection{Single-Index (Sharpe)}
    Sei $I$ Index
    \begin{formula}
        $$\mu_w = \alpha_w + \beta_w \cdot \mu_I$$
        $$\sigma^{2}_w = \beta^{2}_w \cdot \sigma^{2}_I + \sigma^{2}_{\epsilon_w}$$
    \end{formula} 
    \subsection{Multi-Index}
    Seien $I_k$ Indices mit $k \in \{1, \dots, L\}$
    \begin{formula}
       $$\mu_i = \alpha_i + \sum_{k=1}^{L} \beta_{ik} \cdot \mu_{I_k}$$
       $$\sigma^{2}_i = \sum_{k=1}^{L} ( \beta^{2}_{ik} \cdot \sigma^{2}_{I_k} ) + \sigma^{2}_{\epsilon_i}$$
    \end{formula}
    \section{CAPM}
    Sei $w$ WP und $M$ Markt bestehend aus $w^M_i, i \in \{1, \dots, N\}$\\
    Es gilt: $\beta_i = \frac{cov(\tilde{r}_i, \tilde{r}_M)}{\sigma^{2}_M}$
    \begin{formula}[KML]
        $$\mu_w = r + \frac{\mu_M - r}{\sigma_M} \cdot \sigma_w$$
    \end{formula}
    \begin{formula}[WPML]
        $$\mu_j = r + (\mu_M - r) \cdot \beta_i$$
    \end{formula}
    \begin{formula}[Risiko Marktportfolio]
        $$\sigma^{2}_M = \sum_{i=1}^{N}w_i^M \cdot cov(\tilde{r}_i, \tilde{r}_M)$$
    \end{formula}
    \section{APT}
    Seien $I_k, k \in \{1, \dots, K\}$ Faktoren
    \begin{formula}[Eigenschaften]
        $$\sum_{i=1}^{N} w_i = w^T \cdot 1_{N} = 0$$
        $$\sum_{i=1}^{N} w_i \cdot \beta_{i,k} = w^T \cdot \beta_k = 0, \forall k \in \{1, \dots, K\}$$
        $$\sum_{i=1}^{N} w^{2}_i \sigma^{2}_{\epsilon_i} \approx 0$$
    \end{formula}
    \begin{formula}[Linearität]
        $$\exists \lambda_i, i \in \{0, \dots, K\}:  \mu_i = \lambda_0 + \sum_{k=1}^{K}\lambda_k \cdot \beta_{i,k}$$
        Wobei $\lambda_0 = r$ \& $\lambda_k = \mu_{I_k} -r$
    \end{formula}
    \section{Performancemessung}
    \subsection{Sharpe-Ratio}
    \begin{formula}[Sharpe-Ratio]
        $$s = \frac{\bar{\mu} - r}{\sigma}$$
    \end{formula}
    \subsection{Jensen's Alpha}
    \begin{formula}[Jensen's Alpha]
        $$\alpha = \bar{\mu} - r - (\mu_M - r)  \cdot \beta$$
    \end{formula}
    \subsection{Treynor-Maß}
    \begin{formula}[Treynor-Maß]
        $$z = \frac{\bar{\mu} - r}{\beta}$$
    \end{formula}
    \section{Renten}
    \subsection{Zerobonds}
    \begin{formula}[Diskontfaktor Zerobond]
        $$b(0,T) = \frac{P(0)}{RZK}$$
    \end{formula}
    \begin{formula}[Kassazinssätze]
        $$b(0,T) \cdot (1+y(0, T))^T = 1$$
        $$\Leftrightarrow \sqrt[T]{\frac{1}{b(0,T)}} -1 = y(0,T)$$
    \end{formula}
    \subsection{Kuponanleihen}
    \begin{formula}[Emissionspreis]
        $$P(0) = \st(C \cdot (1+y(0,t))^{-t})  + RZK \cdot (1 + y(0,T))^{-T} = \st(C \cdot b(0,t)) + RZK \cdot b(0,T)$$
    \end{formula}
    \begin{formula}[Yield to maturity]
        $$P(0) = \st(C \cdot (1 + y^{*})^{-t})  + RZK \cdot (1 + y^{*})^{-T}$$
        $$\Rightarrow \st(C \cdot (1 + y^{*})^{-t})  + RZK \cdot (1 + y^{*})^{-T} = \st(C \cdot (1+y(0,t))^{-t})  + RZK \cdot (1 + y(0,T))^{-T}$$
    \end{formula}
    \begin{formula}[Termingeschäfte]
        $$(1 + y(0,S))^S \cdot (1 + f(0,S,T))^{T-S} = (1 + y(0,T))^T$$
        $$\Rightarrow f(0,S,T) = \sqrt[T-S]{\frac{(1 + y(0,T))^T}{(1 + y(0,S))^S}} - 1$$
    \end{formula}
    \begin{formula}[Swapsatz]
        $$s(0,T) = \frac{1 - b(0,T)}{\sum_{t=1}^{T}b(0,t)}$$
    \end{formula}
    \begin{formula}[Macauly Duaration]
        $$D_{mac} = \frac{((\sum_{t=1}^{T} t \cdot \frac{c}{(1 + \bar{y})^t}) + T \cdot \frac{RZK}{(1 + \bar{y})^T})}{P}$$  
    \end{formula}
    \begin{formula}[Immunisierungszeitpunkt]
        $$ S = D_{mac}$$
    \end{formula}
\end{document}
